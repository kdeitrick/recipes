\setHeadlines
{
}

\begin{recipe}
[ % Optionale Eingaben
%    preparationtime = {1]{h}},
%    portion = \portion{5},
    source = Taste of Thai Express during first time running an accelerator,
	%https://carlsbadcravings.com/thai-red-curry-chicken-recipe/
]
{Thai Red Curry}

    \ingredients
    {
		1$\sfrac{1}{2}$ & Tbsp olive oil \\
		1 & lb chicken, chopped \\
		$\sfrac{1}{4}$ & cup red curry paste \\
		2 & large white onions, 2" thin strips \\
		2 & red bell peppers, 2" thin strips \\
		 & \\
		4 & tsp ginger, grated \\
		8 & garlic cloves, minced \\
		 & \\
		27 & oz can of coconut milk \\
		2 & Tbsp cornstarch \\
		2 & Tbsp Thai Sweet Chili sauce (Mae Ploy) \\
		1 & Tbsp chili paste \\
		$\sfrac{1}{4}$ & cup soy sauce \\
		$\sfrac{1}{4}$ & cup fish sauce \\
		$\sfrac{1}{4}$ & cup lime juice \\
		2 & Tbsp brown sugar \\
		2 & bay leaves \\
		2 & tsp Thai basil \\
		$\sfrac{1}{2}$ & tsp salt \\
		$\sfrac{1}{2}$ & tsp pepper \\
    }
    
    \preparation
    {
        \step Heat oil over medium high heat in large nonstick skillet. Add onion, bell pepper, and red curry paste and cook just until chicken is no longer pink and desired vegetable tenderness is reached. 
		\step Add ginger and garlic and saute 1 minute.
		\\
		\step Add half of the coconut milk. Mix remaining coconut milk with cornstarch and add to skillet with all remaining ingredients.
		\step Bring to a boil, then reduce to a simmer for 5 minutes or until the sauce thickens and the vegetables reach desired tenderness. 
		\step Discard bay leaf and serve with rice. 
    }
	
	\hint
	{
		2 cups dried rice makes correct amount needed. 
	}

\end{recipe}

%Shawn: how to run meetings:
%@ 8 am: make summary slide from atlis/elog
%doug higgenbothom: touch base, sad committee runs 1:30, 
%take pictures and a line

