\setHeadlines
{
}

\begin{recipe}
[ % Optionale Eingaben
%    preparationtime = {1]{h}},
%    portion = \portion{5},
    source = David's ATK Subscription via Elise,
%	https://www.cookscountry.com/recipes/8227-french-toast-casserole
]
{French Toast Casserole}

    \ingredients
    {
		6 & Tbsp unsalted butter, separated \\
		 & \\
		160 & g brown sugar \\
		1 & Tbsp ground cinnamon \\
		$\sfrac{1}{2}$ & tsp ground nutmeg \\
		$\sfrac{1}{8}$ & tsp salt \\
		 & \\
		16 & slices potato sandwich bread \\
		 & \\
		2$\sfrac{1}{2}$ & cups milk \\
		6 & eggs \\
    }
    
    \preparation
    {
        \step Adjust oven rack to middle position and heat oven to 350$^{\circ}$F degrees. Lightly grease 9"x~13" pan with butter. 
		\step Melt 4$\sfrac{1}{2}$~Tbsp of butter. Mix brown sugar, cinnamon, nutmeg, and salt together in a bowl. 
		\step Sprinkle 3~Tbsp of brown sugar mixture evenly over bottom of prepared dish. Place 5~slices (use bread heels here) in even layer in bottom of dish. Brush bread with 1$\sfrac{1}{2}$~Tbsp melted butter and sprinkle with 3~Tbsp sugar mixture.
		\step Place 5~slices in single layer over first layer, brush with $\sfrac{1}{2}$~Tbsp melted butter, then sprinkle 3~Tbsp sugar mixture. 
		\step Place 6~slices over previous layer and brush with remaining melted butter. 
		\step In separate bowl, whisk milk and eggs together until well combined. Pour milk mixture over bread and press lightly to submerge. Sprinkle with remaining sugar mixture. 
		\step Bake until casserole is slightly puffed, golden brown, and bubbling around the edges, about 30~minutes. 
		\step Transfer casserole to wire rack, brush with 1$\sfrac{1}{2}$~Tbsp melted butter, and let cool for 15~minutes. 
    }

	\suggestion[To Make Ahead]
	{
		The assembled casserole, minus the remaining sugar mixture on top after submerging, can be covered and refrigerated for up to 12~hours. When ready to cook, sprinkle with sugar mixture and proceed to bake as directed. 
	}

	\hint
	{
		Developed with Martin's Potato Bread, and originally 18 slices.
	}

\end{recipe}
