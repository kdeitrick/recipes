\setHeadlines
{
}

\begin{recipe}
[ % Optionale Eingaben
%    preparationtime = {1]{h}},
%    portion = \portion{5},
    source = Mom,
%	
]
{Mom's Pumpkin Bread}

    \ingredients
    {
		1 & cup oil (add a little extra) \\
		3 & cups granulated sugar \\
		 & (594~g) \\
		4 & eggs \\
		15 & oz solid pack pumpkin \\
		$\sfrac{5}{8}$ & cup water \\
		 & (142~g) \\
		 & \\
		$3\sfrac{1}{2}$ & cups AP flour \\
		 & (420~g) \\
		2 & tsp cinnamon \\
		1$\sfrac{1}{2}$ & tsp ginger \\
		1$\sfrac{1}{2}$ & tsp nutmeg \\
		1$\sfrac{1}{2}$ & tsp all-spice \\
		1$\sfrac{1}{2}$ & salt \\
		2 & tsp baking soda \\
    }
    
    \preparation
    {
        \step Preheat the oven to 350$^{\circ}$F. Lightly grease and flour two loaf pans.
		\step In a large bowl, beat together the oil, sugar, eggs, pumpkin, and water. 
		\step Add the rest of the ingredients, stirring to combine well. 
		\step Divide between the two loaf pans. \\
		\step Bake the bread for 60 minutes, or until a cake tester or toothpick inserted in the center comes out clean. 
		\step Remove it from the pan and cool completely on a rack. 
		\step Store leftover bread, tightly wrapped, at room temperature for several days. Freeze for longer storage. 
    }
	
	\suggestion
	{
		Original recipe card had all ingredients combined at once, but I've read too many recipes where dry ingredients are separately combined and added later. 
	}

\end{recipe}
