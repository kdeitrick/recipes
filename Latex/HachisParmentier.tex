\setHeadlines
{
}

\begin{recipe}
[ % Optionale Eingaben
    preparationtime = {An age},
%    portion = \portion{5},
    source = Around my french table by Dorie Greenspan,
]
{Hachis Parmentier}
    
    \ingredients
    {
		\multicolumn{2}{l}{\textbf{For beef and bouillon}} \\
		1 & lb cube steak or boneless beef chuck, cut into small pieces \\
		1 & white onion, sliced \\
		1 & lb carrots, peeled, trimmed, and cut into 1-inch pieces \\
		2 & garlic cloves, smashed and peeled \\
		1 & bay leaf \\
		1 & tsp salt \\
		$\sfrac{1}{4}$ & tsp black peppercorns \\
		6 & cups water \\
		$\sfrac{1}{2}$ & beef bouillon cube \\
		\multicolumn{2}{l}{\textbf{For the filling}} \\
		1$\sfrac{1}{2}$ & Tbsp olive oil \\
		$\sfrac{1}{2}$ & lb sweet sausage, removed from casings if necessary \\
		1 & tsp tomato paste \\
		dash & salt and pepper \\
		\multicolumn{2}{l}{\textbf{For the topping}} \\
		2 & lb red or gold potatoes, peeled and quartered \\
		1 & cup heavy cream \\
		4 & Tbsp unsalted butter \\
		dash & salt and pepper \\
		$\sfrac{1}{2}$ & cup shredded gruyere \\
    }
    
    \preparation
    {
        \step \textbf{To make the beef:} Put all the ingredient except the bouillon cube in a Dutch oven or soup pot and bring to a boil, skimming off the foam and solids that bubble to the surface. Lower the heat and simmer gently for 1$\sfrac{1}{2}$~hours. Taste broth at 45 minutes and add bouillon cube for more intense flavor if desired. 
		\step Drain the meat and carrots, reserving the broth. Strain the broth. (The beef and bouillon can be made up to one day ahead, covered and refrigerated.)
		\step Chop the beef and carrots into tiny pieces using a chef's knife. You could also do this using a food processor.
		\step \textbf{To make the filling:} Butter a 2-quart oven-safe casserole dish.
		\step Put a large skillet over medium heat and pour in olive oil. When it's hot, add sausage and cook, breaking up clumps of meat, until the sausage is just pink. 
		\step Add the chopped beef and tomato paste and stir to mix everything well. 
		\step Stir in 1~cup of the bouillon and bring to a boil. You want just enough bouillon in the pan to moisten the filling and bubble up gently wherever there's a little room; if you think you need more, add it now. Season with salt and pepper.
		\step Add chopped carrots.
		\\
		\step Scrape the filling into the casserole dish and cover it lightly; set aside while you prepare the potatoes. (You can make up to this point a few hours ahead; cover with foil and refrigerate.)
		\step \textbf{To make the topping:} Put the potatoes in a large pot of generously salted cold water and bring to a boil. Cook until potatoes are tender enough to be pierced easily, about 20 minutes; drain well. 
		\step Mash potatoes well, then blend in 3 tablespoons of the butter. Blend in heavy cream to taste. Season to taste with salt and pepper.
		\step Spoon potatoes over filling, spreading them evenly and making sure they reach to the edges of the casserole. Sprinkle shredded gruyere over the top and scatter bits of 1 tablespoon of butter.
		\step Preheat oven to 400$^{\circ}$~F and place drip catcher (silicone baking mat, aluminum foil) under center rack. Place dish on center rack and bake for 30 minutes, or until filling is bubbling steadily and potatoes have developed a golden brown crust.
    }
	
	\suggestion
	{
		For a quicker recipe, use ground beef and store-bought beef broth. Use 1 pound ground beef, and when you add it to the sausage in the skillet, sprinkle a little parsley and thyme. Saute 2 minced garlic cloves in olive oil before adding sausage to skillet. Moisten the filling with broth, and you're good to go. 
	}

\end{recipe}
